\documentclass{article}
\usepackage{amsmath,amssymb}
\usepackage{hyperref}
\usepackage{color}
\title{Notes on ``Finite-Dimensional Vector Spaces''\\by Paul R. Halmos}
\author{}
\begin{document}
\maketitle
Each \verb|\section| corresponds to the scope of one member's assignment,
and each \verb|\subsection| corresponds to one theorem or exercise in the textbook,
specified in the format $m.n$
where $m$ is the section number and $n$ is the theorem/exercise number.
If $n$ is not given, we use $n = 1$ instead.
\section{Toga (2022/09/19)}
\subsection{Exercise 1.1}
\begin{itemize}
  \item[(a)] Since addition is commutative, $0 + \alpha = \alpha + 0$ holds.
    We also have $\alpha + 0 = \alpha$ by definition,
    hence $0 + \alpha = \alpha$.
\end{itemize}
\section{Mohehe}
\subsection{Exercise 1.1}
\begin{itemize}
  \item[(b)]We have $(\alpha + \beta) + (-\alpha) = (\beta + \alpha) + (-\alpha) = \beta + (\alpha + (-\alpha)) = \beta + 0 = \beta$ by definition.
    We also have $(\alpha + \beta) + (-\alpha) = (\alpha + \gamma) + (-\alpha) = (\gamma + \alpha) + (-\alpha) = \gamma + (\alpha + (-\alpha)) = \gamma + 0 = \gamma$ by definition.
    Therefore, $\beta = \gamma$ holds. 
    
    {\color{red}
      If $\alpha + \beta = \alpha + \gamma$, we have $\beta = \beta + 0 = 0 + \beta = (\alpha + (-\alpha)) + \beta = (-\alpha + \alpha) + \beta = -\alpha + (\alpha + \beta) = -\alpha + (\alpha + \gamma) = (-\alpha + \alpha) + \gamma = (\alpha + (-\alpha)) + \gamma = 0 + \gamma = \gamma + 0 = \gamma$ by definition.
    }

  \item[(c)]We have $\alpha + (\beta - \alpha) = \alpha + (\beta + (-\alpha)) = \alpha + ((-\alpha) + \beta) = (\alpha + (-\alpha)) + \beta = 0 + \beta = \beta + 0 = \beta$ by definition.
    
    {\color{red}
      We have $\alpha + (\beta - \alpha) = (\beta - \alpha) + \alpha = \beta + (-\alpha + \alpha) = \beta + (\alpha - \alpha) = \beta + 0 = \beta$ by definition.
    }
    
  \item[(d)]We have $0\cdot\alpha = \alpha\cdot0 = \alpha(1 + (-1)) = \alpha1 + \alpha(-1) = \alpha + (-1)\alpha = \alpha + (-\alpha) = 0$ by definition and Exercise 1(e),
    hence $\alpha\cdot0 = 0\cdot\alpha = 0$.
    
    {\color{red}
      We have $\alpha 0 + \alpha 0 = \alpha (0 + 0) = \alpha 0 = \alpha 0 + 0$ by definition, hence $\alpha 0 = 0$ by Exercise 1(b).
      Note that $0 \alpha = \alpha 0$ by definition.
    }
    
  \item[(e)]
    
    {\color{red}
      We have $\alpha + (-1)\alpha = 1\alpha + (-1)\alpha = (1 + (-1))\alpha = 0\alpha = 0$ by definition and Exercise 1(d).
      Since the additive inverse is unique, we obtain $(-1)\alpha = -\alpha$.
    }
    
  \item[(f)]We have $(-\alpha)(-\beta) = ((-1)\alpha)((-1)\beta) = (\alpha(-1))((-1)\beta) = \alpha((-1)((-1)\beta)) = \alpha((-1)(-1)\beta)$ by Exercise 1(e) and definition.
    We also have $(-1)(-1) = 0 + (-1)(-1) = (1 + (-1)) + (-1)(-1) = 1 + (-1) + (-1)(-1) = 1 + (-1)((-1) + 1) = 1 + (-1)(1 + (-1)) = 1 + (-1)0 = 1 + 0 = 1$ by definition.
    By it and definition, $\alpha((-1)(-1)\beta) = \alpha(1\beta) = \alpha(\beta1) = \alpha\beta$ holds.
    Therefore, $(-\alpha)(-\beta) = \alpha\beta$ holds.
    
  \item[(g)]If $\alpha\beta = 0$,
    suppose $\alpha \neq 0$ and $\beta \neq 0$ hold.
    By supposition and definition, we have $0 = \alpha^{-1}0 = \alpha^{-1}(\alpha\beta) = (\alpha^{-1}\alpha)\beta = (\alpha\alpha^{-1})\beta = 1\beta = \beta1 = \beta$,
    hence $\beta = 0$.
    However, this result contradicts supposition, "$\alpha \neq 0$ and $\beta \neq 0$".
    Therefore, if $\alpha\beta = 0$, then either $\alpha = 0$ or $\beta = 0$ (or both).
    
\end{itemize}
\section{Joh (2022/09/19)}
\subsection{Exercise 1.2}
\begin{itemize}
  \item[(a)] The set of positive integers is not a field since there is no additive inverse for 1.
  \item[(b)] The set of integers is not a field since there is no multiplicative inverse for 2.
  \item[(c)] There exists a bijective map $\varphi$ from $\mathbb{N}$ (or $\mathbb{Z}$) to $\mathbb{Q}$ \cite{Q_countable}, where $\mathbb{Q}$ is a field \cite{Q_field}.
  	We can make $\mathbb{N}$ a field by re-defining (i) addition by $a \oplus b = \varphi^{-1} (\varphi(a) + \varphi(b))$ and (ii) multiplication by $a \otimes b = \varphi^{-1}(\varphi(a)\varphi(b))$ for each $a, b \in \mathbb{N}$.
	Note that the additive and multiplicative identities become $\varphi^{-1}(0)$ and $\varphi^{-1}(1)$, respectively.
	For each $\alpha\in\mathbb{N}$, the additive inverse becomes $\varphi^{-1}(-\varphi(\alpha))$, and the multiplicative inverse becomes $\varphi^{-1}(1/\varphi(\alpha))$ if $\alpha\ne\varphi^{-1}(0)$.
	
	Let $\alpha,\beta,\gamma,\alpha',\beta'\in\mathbb{N}$.
	Note that\\
         1) $\alpha \oplus \beta = \varphi^{-1} (\varphi(\alpha) + \varphi(\beta)) = \varphi^{-1} (\varphi(\beta) + \varphi(\alpha)) = \beta \oplus \alpha$ holds.(addition is commutative)\\
         {\color{red}(from here, mohehe)}\\
         2) $\alpha \oplus (\beta \oplus \gamma) = \alpha \oplus (\varphi^{-1}(\varphi(\beta) + \varphi(\gamma))) = \varphi^{-1}(\varphi(\alpha) + \varphi(\varphi^{-1}(\varphi(\beta) + \varphi(\gamma))))  = \varphi^{-1}(\varphi(\alpha) + (\varphi(\beta) + \varphi(\gamma))) = \varphi^{-1}((\varphi(\alpha) + \varphi(\beta)) + \varphi(\gamma)) = \varphi^{-1}(\varphi(\varphi^{-1}(\varphi(\alpha) + \varphi(\beta))) + \varphi(\gamma)) = \varphi^{-1}(\varphi(\alpha) + \varphi(\beta)) \oplus \gamma = (\alpha \oplus \beta) \oplus \gamma$ holds.(addition is associative)\\
         3) $\alpha \oplus \varphi^{-1}(0) = \varphi^{-1}(\varphi(\alpha) + \varphi(\varphi^{-1}(0))) = \varphi^{-1}(\varphi(\alpha) + 0) = \varphi^{-1}(\varphi(\alpha)) = \alpha$ holds.(there exists additive identity, $\varphi^{-1}(0)$)
         If $\alpha'$ and $\beta'$ are additive identity, 
         we have $\alpha' = \alpha' \oplus \beta' = \beta' \oplus \alpha' = \beta'$ by 1) and the definition of additive identity.(additive identity is unique)\\
         4) $-\varphi(\alpha)\in\mathbb{Q}$ holds by definition, so $\varphi^{-1}(-\varphi(\alpha)) \in \mathbb{N}$ holds. 
         Therefore, $\alpha \oplus \varphi^{-1}(-\varphi(\alpha)) = \varphi^{-1}(\varphi(\alpha) + \varphi(\varphi^{-1}(-\varphi(\alpha)))) = \varphi^{-1}(\varphi(\alpha) +  (-\varphi(\alpha))) = \varphi^{-1}(0)$ holds.(for each $\alpha$ ($\alpha\in\mathbb{N}$), there exists additive inverse)
         For each $\alpha$, if $\alpha'$ and $\beta'$ are additive inverse,
         we have $\alpha' = \alpha' \oplus \varphi^{-1}(0) = \alpha' \oplus (\alpha \oplus \beta') = (\alpha' \oplus \alpha) \oplus \beta' = (\alpha \oplus \alpha') \oplus \beta' = \varphi^{-1}(0) \oplus \beta' = \beta \oplus \varphi^{-1}(0) = \beta'$ by 1), 2), 3) and the definition of additive inverse.(additive inverse is unique)\\
         5) $\alpha \otimes \beta = \varphi^{-1}(\varphi(\alpha)\varphi(\beta)) = \varphi^{-1}(\varphi(\beta)\varphi(\alpha)) = \beta \otimes \alpha$ holds.(multiplication is commutative)\\
         6) $\alpha \otimes (\beta \otimes \gamma) = \alpha \otimes (\varphi^{-1}(\varphi(\beta)\varphi(\gamma))) = \varphi^{-1}(\varphi(\alpha)\varphi(\varphi^{-1}(\varphi(\beta)\varphi(\gamma)))) = \varphi^{-1}(\varphi(\alpha)\varphi(\beta)\varphi(\gamma)) = \varphi^{-1}(\varphi(\varphi^{-1}(\varphi(\alpha)\varphi(\beta)))\varphi(\gamma)) = \varphi^{-1}(\varphi(\alpha)\varphi(\beta)) \otimes \gamma = (\alpha \otimes \beta) \otimes \gamma$ holds.(multiplication is associative)\\
         7) $\alpha \otimes \varphi^{-1}(1) = \varphi^{-1}(\varphi(\alpha)\varphi(\varphi^{-1}(1))) = \varphi^{-1}(\varphi(\alpha)\cdot1) = \alpha$ holds.(there exists additive identity, $\varphi^{-1}(1)$)
         If $\alpha'$ and $\beta'$ are additive identity,
         we have $\alpha' = \alpha' \otimes \beta' = \beta' \otimes \alpha' = \beta'$ by 5) and definition of multiplicative identity.(multiplicative identity is unique)\\
         8) For each $\alpha$ $(\alpha \neq \varphi^{-1}(0))$, $(1/\varphi(\alpha))\in\mathbb{Q}$ holds by definition, so $\varphi^{-1}(1/\varphi(\alpha)) \in \mathbb{N}$ holds. Therefore, $\alpha \otimes \varphi^{-1}(1/\varphi(\alpha)) = \varphi^{-1}(\varphi(\alpha)\varphi(\varphi^{-1}(1/\varphi(\alpha)))) = \varphi^{-1}(\varphi(\alpha)(1/\varphi(\alpha))) = \varphi^{-1}(1)$ holds.(for each $\alpha$ ($\alpha \in \mathbb{N}$), there exists multiplicative inverse)
         For each $\alpha$ ($\alpha \neq \varphi^{-1}(0)$), if $\alpha'$ and $\beta'$ are multiplicative inverse,
         we have $\alpha' = \alpha' \otimes \varphi^{-1}(1) = \alpha' \otimes (\alpha \otimes \beta') = (\alpha' \otimes \alpha) \otimes \beta' = (\alpha \otimes \alpha') \otimes \beta' = \varphi^{-1}(1) \otimes \beta' = \beta' \otimes \varphi^{-1}(1) = \beta'$ by 5), 6), 7) and the definition of multiplicative inverse.(multiplicative inverse is unique) \\
         9) $\alpha \otimes (\beta \oplus \gamma) = \alpha \otimes (\varphi^{-1}(\varphi(\beta) + \varphi(\gamma))) = \varphi^{-1}(\varphi(\alpha)\varphi(\varphi^{-1}(\varphi(\beta) + \varphi(\gamma)))) = \varphi^{-1}(\varphi(\alpha)(\varphi(\beta) + \varphi(\gamma))) = \varphi^{-1}(\varphi(\alpha)\varphi(\beta) + \varphi(\alpha)\varphi(\gamma)) = \varphi^{-1}(\varphi(\varphi^{-1}(\varphi(\alpha)\varphi(\beta))) + \varphi(\varphi^{-1}(\varphi(\alpha)\varphi(\gamma)))) = \varphi^{-1}(\varphi(\alpha \otimes \beta) + \varphi(\alpha \otimes \gamma)) = \alpha \otimes \beta \oplus \alpha \otimes \gamma$ holds.(distributive law stands)
\end{itemize}

\begin{thebibliography}{9}
\bibitem{Q_countable} \url{https://proofwiki.org/wiki/Rational_Numbers_are_Countably_Infinite}
\bibitem{Q_field} \url{https://proofwiki.org/wiki/Rational_Numbers_form_Field}
\end{thebibliography}
\end{document}
