\documentclass{article}
\usepackage{amsmath,amssymb}
\usepackage{hyperref}
\title{Notes on ``Finite-Dimensional Vector Spaces''\\by Paul R. Halmos}
\author{}
\begin{document}
\maketitle
Each \verb|\section| corresponds to the scope of one member's assignment,
and each \verb|\subsection| corresponds to one theorem or exercise in the textbook,
specified in the format $m.n$
where $m$ is the section number and $n$ is the theorem/exercise number.
If $n$ is not given, we use $n = 1$ instead.
\section{Toga (2022/09/19)}
\subsection{Exercise 1.1}
\begin{itemize}
  \item[(a)] Since addition is commutative, $0 + \alpha = \alpha + 0$ holds.
    We also have $\alpha + 0 = \alpha$ by definition,
    hence $0 + \alpha = \alpha$.
\end{itemize}
\section{Mohehe}
\subsection{Exercise 1.1}
\begin{itemize}
  \item[(b)]
  \item[(c)]
  \item[(d)]
  \item[(e)]
  \item[(f)]
  \item[(g)]
\end{itemize}
\section{Joh (2022/09/19)}
\subsection{Exercise 1.2}
\begin{itemize}
  \item[(a)] The set of positive integers is not a field since there is no additive inverse for 1.
  \item[(b)] The set of integers is not a field since there is no multiplicative inverse for 2.
  \item[(c)] There exists a bijective map $\varphi$ from $\mathbb{N}$ (or $\mathbb{Z}$) to $\mathbb{Q}$ \cite{Q_countable}, where $\mathbb{Q}$ is a field \cite{Q_field}.
  	We can make $\mathbb{N}$ a field by re-defining (i) addition by $a \oplus b = \varphi^{-1} (\varphi(a) + \varphi(b))$ and (ii) multiplication by $a \otimes b = \varphi^{-1}(\varphi(a)\varphi(b))$ for each $a, b \in \mathbb{N}$.
	Note that the additive and multiplicative identities become $\varphi^{-1}(0)$ and $\varphi^{-1}(1)$, respectively.
	For each $\alpha\in\mathbb{N}$, the additive inverse becomes $\varphi^{-1}(-\varphi(\alpha))$, and the multiplicative inverse becomes $\varphi^{-1}(1/\varphi(\alpha))$ if $\alpha\ne\varphi^{-1}(0)$.
\end{itemize}

\begin{thebibliography}{9}
\bibitem{Q_countable} \url{https://proofwiki.org/wiki/Rational_Numbers_are_Countably_Infinite}
\bibitem{Q_field} \url{https://proofwiki.org/wiki/Rational_Numbers_form_Field}
\end{thebibliography}
\end{document}
