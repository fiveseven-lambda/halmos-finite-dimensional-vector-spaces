\documentclass{article}
\usepackage{amsmath,amssymb}
\usepackage{hyperref}
\usepackage{color}
\usepackage{enumerate}

\newcommand\Z{\mathbb Z}

\title{Notes on ``Finite-Dimensional Vector Spaces''\\by Paul R. Halmos}
\author{}
\begin{document}
\maketitle
Each \verb|\section| corresponds to the scope of one member's assignment,
and each \verb|\subsection| corresponds to one theorem or exercise in the textbook,
specified in the format $m.n$
where $m$ is the section number and $n$ is the theorem/exercise number.
If $n$ is not given, we use $n = 1$ instead.
\section{Toga (2022/09/19)}
\subsection{Exercise 1.1}
\begin{itemize}
  \item[(a)] Since addition is commutative, $0 + \alpha = \alpha + 0$ holds.
    We also have $\alpha + 0 = \alpha$ by definition,
    hence $0 + \alpha = \alpha$.
\end{itemize}
\section{Mohehe (2022/09/27)}
\subsection{Exercise 1.1}
\begin{itemize}
  \item[(b)]If $\alpha + \beta = \alpha + \gamma$, 
    we have $\beta = \beta + 0 = 0 + \beta = (\alpha + (-\alpha)) + \beta = ((-\alpha) + \alpha) + \beta = (-\alpha) + (\alpha + \beta) = (-\alpha) + (\alpha + \gamma) = ((-\alpha) + \alpha) + \gamma = (\alpha + (-\alpha)) + \gamma = 0 + \gamma = \gamma + 0 = \gamma$ by definition.
    Therefore, $\beta = \gamma$ holds.

  \item[(c)]We have $\alpha + (\beta - \alpha) = \alpha + (\beta + (-\alpha)) = \alpha + ((-\alpha) + \beta) = (\alpha + (-\alpha)) + \beta = 0 + \beta = \beta + 0 = \beta$ by definition.
    Therefore, $\alpha + (\beta - \alpha) = \beta$ holds.
    
  \item[(d)]We have $\alpha0 + \alpha0 = \alpha(0 + 0) = \alpha0 = \alpha + 0$ by definition,
    hence $\alpha0 = 0$ by Exercise 1(b).
    We also have $\alpha\cdot0 = 0\cdot\alpha$ by definition.
    Therefore, $\alpha\cdot0 = 0\cdot\alpha = 0$
    
  \item[(e)]We have $\alpha + (-1)\alpha = 1\alpha + (-1)\alpha = (1 + (-1))\alpha = 0\alpha = 0$ by definition and Exercise 1(d).
      Since the additive inverse is unique,
      we obtain $(-1)\alpha = -\alpha$.
    
  \item[(f)]We have $(-\alpha)(-\beta) = ((-1)\alpha)((-1)\beta) = (\alpha(-1))((-1)\beta) = \alpha((-1)((-1)\beta)) = \alpha((-1)(-1)\beta)$ by Exercise 1(e) and definition.
    We also have $(-1)(-1) = 0 + (-1)(-1) = (1 + (-1)) + (-1)(-1) = 1 + (-1) + (-1)(-1) = 1 + (-1)((-1) + 1) = 1 + (-1)(1 + (-1)) = 1 + (-1)0 = 1 + 0 = 1$ by definition.
    By it and definition, $\alpha((-1)(-1)\beta) = \alpha(1\beta) = \alpha(\beta1) = \alpha\beta$ holds.
    Therefore, $(-\alpha)(-\beta) = \alpha\beta$ holds.
    
  \item[(g)]If $\alpha\beta = 0$,
    suppose $\alpha \neq 0$ and $\beta \neq 0$ hold.
    By supposition and definition, we have $0 = \alpha^{-1}0 = \alpha^{-1}(\alpha\beta) = (\alpha^{-1}\alpha)\beta = (\alpha\alpha^{-1})\beta = 1\beta = \beta1 = \beta$,
    hence $\beta = 0$.
    However, this result contradicts supposition, "$\alpha \neq 0$ and $\beta \neq 0$".
    Therefore, if $\alpha\beta = 0$, then either $\alpha = 0$ or $\beta = 0$ (or both).
    
\end{itemize}
\section{Joh (2022/09/19)}
\subsection{Exercise 1.2}
\begin{itemize}
  \item[(a)] The set of positive integers is not a field since there is no additive inverse for 1.
  \item[(b)] The set of integers is not a field since there is no multiplicative inverse for 2.
  \item[(c)] There exists a bijective map $\varphi$ from $\mathbb{N}$ (or $\mathbb{Z}$) to $\mathbb{Q}$ \cite{Q_countable}, where $\mathbb{Q}$ is a field \cite{Q_field}.
  	We can make $\mathbb{N}$ a field by re-defining (i) addition by $a \oplus b = \varphi^{-1} (\varphi(a) + \varphi(b))$ and (ii) multiplication by $a \otimes b = \varphi^{-1}(\varphi(a)\varphi(b))$ for each $a, b \in \mathbb{N}$.
	Note that the additive and multiplicative identities become $\varphi^{-1}(0)$ and $\varphi^{-1}(1)$, respectively.
	For each $\alpha\in\mathbb{N}$, the additive inverse becomes $\varphi^{-1}(-\varphi(\alpha))$, and the multiplicative inverse becomes $\varphi^{-1}(1/\varphi(\alpha))$ if $\alpha\ne\varphi^{-1}(0)$.
	
	Let $\alpha,\beta,\gamma,\alpha',\beta'\in\mathbb{N}$.
	Note that

\begin{itemize}
         \item[1)] $\alpha \oplus \beta = \varphi^{-1} (\varphi(\alpha) + \varphi(\beta)) = \varphi^{-1} (\varphi(\beta) + \varphi(\alpha)) = \beta \oplus \alpha$ holds.(addition is commutative)\\
         {\color{red}(from here, mohehe)}
         \item[2)] $\alpha \oplus (\beta \oplus \gamma) = \alpha \oplus (\varphi^{-1}(\varphi(\beta) + \varphi(\gamma))) = \varphi^{-1}(\varphi(\alpha) + \varphi(\varphi^{-1}(\varphi(\beta) + \varphi(\gamma))))  = \varphi^{-1}(\varphi(\alpha) + (\varphi(\beta) + \varphi(\gamma))) = \varphi^{-1}((\varphi(\alpha) + \varphi(\beta)) + \varphi(\gamma)) = \varphi^{-1}(\varphi(\varphi^{-1}(\varphi(\alpha) + \varphi(\beta))) + \varphi(\gamma)) = \varphi^{-1}(\varphi(\alpha) + \varphi(\beta)) \oplus \gamma = (\alpha \oplus \beta) \oplus \gamma$ holds.(addition is associative)
         \item[3)] $\alpha \oplus \varphi^{-1}(0) = \varphi^{-1}(\varphi(\alpha) + \varphi(\varphi^{-1}(0))) = \varphi^{-1}(\varphi(\alpha) + 0) = \varphi^{-1}(\varphi(\alpha)) = \alpha$ holds.(there exists additive identity, $\varphi^{-1}(0)$)
  If $\alpha'$ and $\beta'$ are additive identity, we have $\alpha' = \alpha' \oplus \beta' = \beta' \oplus \alpha' = \beta'$ by 1) and the definition of additive identity.(additive identity is unique)
         \item[4)] $-\varphi(\alpha)\in\mathbb{Q}$ holds by definition, so $\varphi^{-1}(-\varphi(\alpha)) \in \mathbb{N}$ holds. 
   Therefore, $\alpha \oplus \varphi^{-1}(-\varphi(\alpha)) = \varphi^{-1}(\varphi(\alpha) + \varphi(\varphi^{-1}(-\varphi(\alpha)))) = \varphi^{-1}(\varphi(\alpha) +  (-\varphi(\alpha))) = \varphi^{-1}(0)$ holds.(for each $\alpha$ ($\alpha\in\mathbb{N}$), there exists additive inverse)
  For each $\alpha$, if $\alpha'$ and $\beta'$ are additive inverse, we have $\alpha' = \alpha' \oplus \varphi^{-1}(0) = \alpha' \oplus (\alpha \oplus \beta') = (\alpha' \oplus \alpha) \oplus \beta' = (\alpha \oplus \alpha') \oplus \beta' = \varphi^{-1}(0) \oplus \beta' = \beta \oplus \varphi^{-1}(0) = \beta'$ by 1), 2), 3) and the definition of additive inverse.(additive inverse is unique)
         \item[5)] $\alpha \otimes \beta = \varphi^{-1}(\varphi(\alpha)\varphi(\beta)) = \varphi^{-1}(\varphi(\beta)\varphi(\alpha)) = \beta \otimes \alpha$ holds.(multiplication is commutative)
         \item[6)] $\alpha \otimes (\beta \otimes \gamma) = \alpha \otimes (\varphi^{-1}(\varphi(\beta)\varphi(\gamma))) = \varphi^{-1}(\varphi(\alpha)\varphi(\varphi^{-1}(\varphi(\beta)\varphi(\gamma)))) = \varphi^{-1}(\varphi(\alpha)\varphi(\beta)\varphi(\gamma)) = \varphi^{-1}(\varphi(\varphi^{-1}(\varphi(\alpha)\varphi(\beta)))\varphi(\gamma)) = \varphi^{-1}(\varphi(\alpha)\varphi(\beta)) \otimes \gamma = (\alpha \otimes \beta) \otimes \gamma$ holds.(multiplication is associative)
         \item[7)] $\alpha \otimes \varphi^{-1}(1) = \varphi^{-1}(\varphi(\alpha)\varphi(\varphi^{-1}(1))) = \varphi^{-1}(\varphi(\alpha)\cdot1) = \alpha$ holds.(there exists additive identity, $\varphi^{-1}(1)$)
  If $\alpha'$ and $\beta'$ are additive identity, we have $\alpha' = \alpha' \otimes \beta' = \beta' \otimes \alpha' = \beta'$ by 5) and definition of multiplicative identity.(multiplicative identity is unique)
         \item[8)] For each $\alpha$ $(\alpha \neq \varphi^{-1}(0))$, $(1/\varphi(\alpha))\in\mathbb{Q}$ holds by definition, so $\varphi^{-1}(1/\varphi(\alpha)) \in \mathbb{N}$ holds. Therefore, $\alpha \otimes \varphi^{-1}(1/\varphi(\alpha)) = \varphi^{-1}(\varphi(\alpha)\varphi(\varphi^{-1}(1/\varphi(\alpha)))) = \varphi^{-1}(\varphi(\alpha)(1/\varphi(\alpha))) = \varphi^{-1}(1)$ holds.(for each $\alpha$ ($\alpha \in \mathbb{N}$), there exists multiplicative inverse)
  For each $\alpha$ ($\alpha \neq \varphi^{-1}(0)$), if $\alpha'$ and $\beta'$ are multiplicative inverse, we have $\alpha' = \alpha' \otimes \varphi^{-1}(1) = \alpha' \otimes (\alpha \otimes \beta') = (\alpha' \otimes \alpha) \otimes \beta' = (\alpha \otimes \alpha') \otimes \beta' = \varphi^{-1}(1) \otimes \beta' = \beta' \otimes \varphi^{-1}(1) = \beta'$ by 5), 6), 7) and the definition of multiplicative inverse.(multiplicative inverse is unique) 
         \item[9)] $\alpha \otimes (\beta \oplus \gamma) = \alpha \otimes (\varphi^{-1}(\varphi(\beta) + \varphi(\gamma))) = \varphi^{-1}(\varphi(\alpha)\varphi(\varphi^{-1}(\varphi(\beta) + \varphi(\gamma)))) = \varphi^{-1}(\varphi(\alpha)(\varphi(\beta) + \varphi(\gamma))) = \varphi^{-1}(\varphi(\alpha)\varphi(\beta) + \varphi(\alpha)\varphi(\gamma)) = \varphi^{-1}(\varphi(\varphi^{-1}(\varphi(\alpha)\varphi(\beta))) + \varphi(\varphi^{-1}(\varphi(\alpha)\varphi(\gamma)))) = \varphi^{-1}(\varphi(\alpha \otimes \beta) + \varphi(\alpha \otimes \gamma)) = \alpha \otimes \beta \oplus \alpha \otimes \gamma$ holds.(distributive law stands)
\end{itemize}
\end{itemize}

\section{Mohehe}
\subsection{Exercise 1.3}
\newcommand\rem{\mathbin\%} % 新しいコマンド \rem を定義.
\newcommand\Zm{\mathcal Z_m} % 新しいコマンド \Zm を定義.
For two integers $a$ and $b$,
we denote by $a \rem b$ the remainder after dividing $a$ by $b$,
and write $b \mid a$ if and only if $a \rem b = 0$.
For clarity,
we denote the ordinary sum and product of two integers $a$ and $b$
by $a +_\Z b$ and $a \cdot_\Z b$, respectively.
Note that $\alpha + \beta = (\alpha +_\Z \beta) \rem m$
and $\alpha\beta = (\alpha \cdot_\Z \beta) \rem m$
for $\alpha, \beta \in \Zm$.
\begin{enumerate}[(a)]
  \item  Let $\alpha,\beta,\gamma \in \Zm$, $k \in \mathbb{Z}$
\begin{itemize}
    \item[1'] Proof : if $m$ is a prime, $\Zm$ is a field.\\
  Suppose $m$ is a prime,
\begin{enumerate}[1)]
         \item  $\alpha + \beta = (\alpha +_\Z \beta) \rem m = (\beta +_\Z \alpha) \rem m = \beta + \alpha$ (addition is commutative)
         \item  Since $\alpha +_\Z (\beta + \gamma) = \alpha +_\Z (\beta +_\Z \gamma) \rem m \equiv \alpha +_\Z (\beta +_\Z \gamma) = (\alpha +_\Z \beta) +_\Z \gamma \equiv (\alpha +_\Z \beta) \rem m +_\Z \gamma = (\alpha + \beta) +_\Z \gamma \pmod m$ holds, $\alpha + (\beta + \gamma) = (\alpha +_\Z (\beta + \gamma)) \rem m  = ((\alpha + \beta) +_\Z \gamma) \rem m = (\alpha + \beta) + \gamma $ holds.(addition is associative)
         \item  $\alpha + 0 = (\alpha +_\Z 0) \rem m = \alpha \rem m = \alpha$ (there exists additive identity)  By it and 1), if $\beta$ and $\gamma$ are additive identity, $\beta = \beta + \gamma = \gamma + \beta = \gamma$ (additive identity is unique)
         \item  If $\alpha + _\Z\beta = m$, $\alpha + \beta = (\alpha + _\Z\beta) \rem m = m \rem m = 0$ (there exists additive inverse)
         \item  $\alpha\beta = (\alpha \cdot_\Z \beta) \rem m = (\beta \cdot_\Z \alpha) \rem m = \beta\alpha$ (multiplication is commutative)
         \item  Since $\alpha \cdot_\Z (\beta\gamma) = \alpha \cdot_\Z ((\beta \cdot_\Z \gamma) \rem m) \equiv \alpha \cdot_\Z (\beta \cdot_\Z \gamma) = (\alpha \cdot_\Z \beta) \cdot_\Z \gamma \equiv ((\alpha \cdot_\Z \beta) \rem m) \cdot_\Z \gamma = (\alpha\beta) \cdot_\Z \gamma \pmod m$ holds, $\alpha(\beta\gamma) = (\alpha \cdot_\Z (\beta\gamma)) \rem m = ((\alpha\beta)\cdot_\Z \gamma) \rem m = (\alpha\beta)\gamma$ holds.(multiplication is associative)
         \item  $\alpha1 = (\alpha _\Z1)\rem m = \alpha \rem m = \alpha$ (there exists multiplicative identity)
  By it and 5), if $\beta$ and $\gamma$ are multiplicative identity, $\beta = \beta\gamma = \gamma\beta = \gamma$ (multiplicative identity is unique)
         \item For all $\alpha (\alpha \neq 0)$, suppose there doesn't exist $\beta$ that makes $\alpha\beta = 1$.
  There exist $\beta, \gamma \in \Zm$ with $\beta \neq \gamma$ and $\alpha\beta = \alpha\gamma$, because $\beta$ is any one from 0 to $m-1$ and $\alpha\beta$ is any one from 0 to $m-1$ except 1.
  Therefore, $(\alpha \cdot_\Z \beta +_\Z (- \alpha \cdot_\Z \gamma) =)\; \alpha \cdot_\Z (\beta +_\Z (-\gamma)) = km$ holds.
  The right side has divisor m, but it contradicts that the left side doesn't have divisor of m except 1,  because $0<\alpha<(m-1)$ and ($(-m)<(\beta +_\Z (-\gamma))<0$ or $0<(\beta +_\Z (-\gamma))<m$) holds.
  Thus, there exists $\beta$ that makes $\alpha\beta = 1$.(there exists maltiplicative inverse)
         \item $\alpha(\beta + \gamma) = (\alpha \cdot_\Z (\beta + \gamma))\rem m = (\alpha \cdot_\Z ((\beta +_\Z \gamma)\rem m))\rem m \equiv (\alpha \cdot_\Z (\beta +_\Z \gamma))\rem m = (\alpha \cdot_\Z \beta + \alpha \cdot_\Z \gamma)\rem m \equiv ((\alpha \cdot_\Z \beta)\rem m +_\Z (\alpha \cdot_\Z \gamma)\rem m)\rem m \equiv (\alpha \cdot\_Z \beta)\rem m + (\alpha \cdot_\Z \gamma)\rem m \equiv \alpha\beta + \alpha\gamma$ holds.(distributive law stands)
\end{enumerate}
  In conclusion, if $m$ is a prime, $\Zm$ is a field.
    \item[2']Proof : If $\Zm$ is a field, $m$ is a prime.\\
  By contraposition, it is equivalent to prove ``If $m$ is not a prime, $\Zm$ is not a field.''
  We can prove 1) to 7) and 9) by the same way.
  For all $\alpha(\alpha \neq 0)$, suppose there exists $\beta$ that makes $\alpha\beta = 1$.
  If $m= 4$ and $\alpha = 2$, $\alpha0 = 0, \alpha1 = 2, \alpha2 = 0, \alpha3 = 2$, but it contradicts that there exists $\beta$ that makes $\alpha\beta = 1$.
  Therefore there exists $\alpha$ that doesn't have $\beta$ that makes $\alpha\beta = 1$.
  In conclusion, ``If $m$ is not a prime, $\Zm$ is not a field.'' and ``If $\Zm$ is a field, $m$ is a prime.''
\end{itemize}
  Because of 1' and 2', $\Zm$ is a field if and only if $m$ is a prime.

  \item $4$
  \item $5$
\end{enumerate}

% \rem と \Zm はこの問題でしか使わないので,以下で消す
\let\rem\undefined
\let\Zm\undefined

\subsection{Exercise 1.4}
\newcommand\F{\mathfrak F}
Define $\alpha_0, \alpha_1, \ldots, \in \F$ as below:
\begin{align}
  \alpha_0 &= 0, \\
  \alpha_n &= \alpha_{n - 1} + 1 \qquad (n > 0).
\end{align}
We have $\alpha_m \alpha_n = \alpha_{mn}$ for all $m$ and $n$.
% ここに証明①
Assume there exists an $n$ with $\alpha_n = 0$ but $\alpha_k \ne 0$ for any $k < n$.
It suffices to prove that $n$ is a prime.
% ここに証明②

\begin{thebibliography}{9}
\bibitem{Q_countable} \url{https://proofwiki.org/wiki/Rational_Numbers_are_Countably_Infinite}
\bibitem{Q_field} \url{https://proofwiki.org/wiki/Rational_Numbers_form_Field}
\end{thebibliography}
\end{document}
