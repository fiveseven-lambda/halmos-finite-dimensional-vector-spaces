\documentclass{article}
\usepackage{amsmath,amssymb}
\usepackage{hyperref}
\usepackage{color}
\title{Notes on ``Finite-Dimensional Vector Spaces''\\by Paul R. Halmos}
\author{}
\begin{document}
\maketitle
Each \verb|\section| corresponds to the scope of one member's assignment,
and each \verb|\subsection| corresponds to one theorem or exercise in the textbook,
specified in the format $m.n$
where $m$ is the section number and $n$ is the theorem/exercise number.
If $n$ is not given, we use $n = 1$ instead.
\section{Toga (2022/09/19)}
\subsection{Exercise 1.1}
\begin{itemize}
  \item[(a)] Since addition is commutative, $0 + \alpha = \alpha + 0$ holds.
    We also have $\alpha + 0 = \alpha$ by definition,
    hence $0 + \alpha = \alpha$.
\end{itemize}
\section{Mohehe}
\subsection{Exercise 1.1}
\begin{itemize}
  \item[(b)]Since addition is commutative, $(\alpha + \beta) + (-\alpha) = (\beta + \alpha) + (-\alpha)$ holds.
    We have $(\beta + \alpha) + (-\alpha) = \beta + (\alpha + (-\alpha))$ because addition is associative.
    We obtain $\beta + (\alpha + (-\alpha)) = \beta + 0$ by definition.
    We also have $\beta + 0 = \beta$ because of definition,
    hence $(\alpha + \beta) + (-\alpha) = \beta$.
    Since addition is commutative, $(\alpha + \gamma) + (-\alpha) = (\gamma + \alpha) + (-\alpha)$ holds.
    We have $(\gamma + \alpha) + (-\alpha) = \gamma + (\alpha + (-\alpha))$ because addition is associative.
    We obtain $\gamma + (\alpha + (-\alpha)) = \gamma + 0$ by definition.
    We also have $\gamma + 0 = \gamma$ because of definition,
    thus $(\alpha + \gamma) + (-\alpha) = \gamma$.
    In addition, we have $(\alpha + \beta) + (-\alpha) = (\alpha + \gamma) + (-\alpha)$,
    therefore $\beta = \gamma$.
    
    {\color{red}
      If $\alpha + \beta = \alpha + \gamma$, we have $\beta = \beta + 0 = 0 + \beta = (\alpha + (-\alpha)) + \beta = (-\alpha + \alpha) + \beta = -\alpha + (\alpha + \beta) = -\alpha + (\alpha + \gamma) = (-\alpha + \alpha) + \gamma = (\alpha + (-\alpha)) + \gamma = 0 + \gamma = \gamma + 0 = \gamma$ by definition.
    }
  \item[(c)]We obtain $\alpha + (\beta - \alpha) = \alpha + (\beta + (-\alpha))$ because of the sentence in the problems.
    Since addition is commutative, $\alpha + (\beta + (-\alpha)) = (\beta + (-\alpha)) + \alpha$ holds.
    We have $(\beta + (-\alpha)) + \alpha = \beta + ((-\alpha) + \alpha)$ because addition is associative.
    We obtain $\beta + ((-\alpha) + \alpha) = \beta + (\alpha + (-\alpha))$ because addition is commutive.
    In additon, the definition leads $\beta + (\alpha + (-\alpha)) = \beta + 0$.
    We also have $\beta + 0 = \beta$,
    hence $\alpha + (\beta - \alpha) = \beta$.
    
    {\color{red}
      We have $\alpha + (\beta - \alpha) = (\beta - \alpha) + \alpha = \beta + (-\alpha + \alpha) = \beta + (\alpha - \alpha) = \beta + 0 = \beta$ by definition.
    }
    
  \item[(d)]
    We have $\alpha\cdot(\beta + (-\beta)) = \alpha\cdot0$ by the definition of addition.
    We obtain $\alpha\cdot0 = 0\cdot\alpha$ because multipulication is commutative.
    The definition of multiplication leads $\alpha\cdot(\beta + (-\beta)) = \alpha\beta + \alpha(-\beta)$.
    Since multiplication is commutative, $\alpha\beta + \alpha(-\beta) = \beta\alpha + (-\beta)\alpha$.
    We obtain $\beta\alpha + (-\beta)\alpha = \beta\alpha + (-1)\beta\alpha$ by exercises1.(e)
    We also have $\beta\alpha + (-1)\beta\alpha = \beta\alpha + (-1)(\beta\alpha)$ because multiplication is associative.
    Exercises1.(e) leads $\beta\alpha + (-1)(\beta\alpha) = \beta\alpha + (-\beta\alpha)$.
    We have $\beta\alpha + (-\beta\alpha) = 0$ by the definition of addition,
    hence $\alpha\cdot0 = 0\cdot\alpha = 0$ holds.
    
    {\color{red}
      We have $\alpha 0 + \alpha 0 = \alpha (0 + 0) = \alpha 0 = \alpha 0 + 0$ by definition, hence $\alpha 0 = 0$ by Exercise 1(b).
      Note that $0 \alpha = \alpha 0$ by definition.
    }
    
  \item[(e)]We have $(-1)\alpha = (-\alpha\alpha^{-1})\alpha$ by the definition of multiplication.
    Since multiplication is associative, $(-\alpha\alpha^{-1})\alpha = (-\alpha)(\alpha^{-1}\alpha)$.
    We obtain $(-\alpha)(\alpha^{-1}\alpha) = (-\alpha)1$ by the definition of multiplication.
    We also have $(-\alpha)1 = -\alpha$ by the definition of multiplication, 
    thus $(-1)\alpha = -\alpha$ holds.
    
    {\color{red}
      We have $(-1)\alpha + \alpha = (-1)\alpha + 1\alpha = (-1 + 1)\alpha = (1 - 1)\alpha = 0\alpha = 0$ by definition and Exercise 1(d).
    }
    
  \item[(f)]We have $(-\alpha)(-\beta) = ((-1)(\alpha))((-1)(\beta))$ by exercise.1(e)
    We obtain $((-1)(\alpha))((-1)(\beta)) = ((\alpha)(-1))((-1)(\beta)) = \alpha((-1)((-1)\beta)) = \alpha((-1)(-1)\beta)$ by definition.
    We also have $(-1)(-1) + (1 + (-1)) = (-1)(-1) + ((-1) + 1) = (-1)(-1) + (-1) + 1 = (-1)(-1) + (-1)1 + 1 = (-1)(-1 + 1) + 1 = (-1)(1 + (-1)) + 1 = (-1)0 + 1 = 0 + 1 = 1 + 0 = 1$ by definition,
    thus, $\alpha((-1)(-1)\beta) = \alpha(1\beta)$ holds.
    $\alpha(1\beta) = \alpha(\beta1) = \alpha\beta$ by definition.
    Therefore, $(-\alpha)(-\beta) = \alpha\beta$ holds.
    
  \item[(g)]If $\beta \neq 0$, we have $(\alpha\beta)\beta^{-1}=\alpha(\beta\beta^{-1})$ because multiplication is associative.
    We obtain $\alpha(\beta\beta^{-1}) = \alpha1$ by the definition of multiplication.
    We have $\alpha1 = \alpha$ by the definition of multiplication.
    We obtain $0\cdot\beta^{-1} = 0$ by exercises1(d).
    thus if $\beta \neq 0$, $\alpha = 0$.
    If $\alpha \neq 0$, we have $(\alpha\beta)\alpha^{-1} = (\beta\alpha)\alpha^{-1}$ because multiplication is commutative.
    We obtain $(\beta\alpha)\alpha^{-1} = \beta(\alpha\alpha^{-1})$ because multiplication is associative.
    We have $\beta(\alpha\alpha^{-1}) = \beta1$  by the definition of multiplication.
    We have $\beta1 = \beta$ by the definition of multiplication.
    We obtain $0\cdot\beta^{-1} = 0$ by exercises1(d).
    thus if $\alpha \neq 0$, $\beta = 0$.
    If $\alpha = 0$ and $\beta = 0$, $\alpha\beta = 0$ by exercise1.(d)
    Therefore, If $\alpha\beta = 0$, then either $\alpha = 0$ or $\beta = 0$ (or both)

    (Another way)
    By contraposition, "If $\alpha\beta = 0$, then either $\alpha = 0$ or $\beta = 0$ (or both)." is equivalent to prove "If both of $\alpha$ and $\beta$ are not zero, $\alpha\beta \neq 0$ holds."
    Suppose $\alpha$ and $\beta$ are not zero.
    We have $0 = \alpha0$ by exercise 1.(d)
    In addition, $\alpha0 \neq \alpha\beta$ holds because of $\beta \neq 0$.
    Therefore, if both of $\alpha$ and $\beta$ are not zero, $\alpha\beta \neq 0$ holds.
    Thus, if $\alpha\beta = 0$, then either $\alpha = 0$ or $\beta = 0$ (or both), by proof of contrapositive.
    
\end{itemize}
\section{Joh (2022/09/19)}
\subsection{Exercise 1.2}
\begin{itemize}
  \item[(a)] The set of positive integers is not a field since there is no additive inverse for 1.
  \item[(b)] The set of integers is not a field since there is no multiplicative inverse for 2.
  \item[(c)] There exists a bijective map $\varphi$ from $\mathbb{N}$ (or $\mathbb{Z}$) to $\mathbb{Q}$ \cite{Q_countable}, where $\mathbb{Q}$ is a field \cite{Q_field}.
  	We can make $\mathbb{N}$ a field by re-defining (i) addition by $a \oplus b = \varphi^{-1} (\varphi(a) + \varphi(b))$ and (ii) multiplication by $a \otimes b = \varphi^{-1}(\varphi(a)\varphi(b))$ for each $a, b \in \mathbb{N}$.
	Note that the additive and multiplicative identities become $\varphi^{-1}(0)$ and $\varphi^{-1}(1)$, respectively.
	For each $\alpha\in\mathbb{N}$, the additive inverse becomes $\varphi^{-1}(-\varphi(\alpha))$, and the multiplicative inverse becomes $\varphi^{-1}(1/\varphi(\alpha))$ if $\alpha\ne\varphi^{-1}(0)$.
\end{itemize}

\begin{thebibliography}{9}
\bibitem{Q_countable} \url{https://proofwiki.org/wiki/Rational_Numbers_are_Countably_Infinite}
\bibitem{Q_field} \url{https://proofwiki.org/wiki/Rational_Numbers_form_Field}
\end{thebibliography}
\end{document}
