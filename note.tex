\documentclass{article}
\usepackage{amsmath,amssymb, amsthm}
\usepackage{hyperref}
\usepackage{color}
\usepackage{enumitem}
\allowdisplaybreaks

\theoremstyle{thmstyleone}
\newtheorem{theorem}{Theorem}[section]
\newtheorem{proposition}[theorem]{Proposition}
\theoremstyle{thmstyletwo}
\newtheorem{remark}[theorem]{Remark}
\theoremstyle{thmstylethree}
\newtheorem{definition}[theorem]{Definition}
\newtheorem{lemma}[theorem]{Lemma}
\newtheorem{corollary}[theorem]{Corollary}
\newtheorem*{notation}{Notation}

\newcommand\Z{\mathcal Z}
\newcommand\N{\mathcal N}
\newcommand\Q{\mathcal Q}
\newcommand\R{\mathcal R}
\newcommand\C{\mathcal C}
\DeclareMathOperator \id{id}

\title{Notes on ``Finite-Dimensional Vector Spaces''\\by Paul R. Halmos}
\author{}
\begin{document}
\maketitle
Each \verb|\section| corresponds to the scope of one member's assignment,
and each \verb|\subsection| corresponds to one theorem or exercise in the textbook,
specified in the format $m.n$
where $m$ is the section number and $n$ is the theorem/exercise number.
If $n$ is not given, we use $n = 1$ instead.
\tableofcontents
\section{Toga (2022/09/19)}
\subsection{Exercise 1.1}
\label{sec:ex1.1}
\begin{enumerate}[label = (\alph*)]
  \item Since addition is commutative, $0 + \alpha = \alpha + 0$ holds.
    We also have $\alpha + 0 = \alpha$ by definition,
    hence $0 + \alpha = \alpha$.
\end{enumerate}
\section{Mohehe (2022/09/27)}
\subsection{Exercise 1.1}
\begin{enumerate}[label = (\alph*), resume]
  \item If $\alpha + \beta = \alpha + \gamma$, 
    we have $\beta = \beta + 0 = 0 + \beta = (\alpha + (-\alpha)) + \beta = ((-\alpha) + \alpha) + \beta = (-\alpha) + (\alpha + \beta) = (-\alpha) + (\alpha + \gamma) = ((-\alpha) + \alpha) + \gamma = (\alpha + (-\alpha)) + \gamma = 0 + \gamma = \gamma + 0 = \gamma$ by definition.
    Therefore, $\beta = \gamma$ holds.

  \item We have $\alpha + (\beta - \alpha) = \alpha + (\beta + (-\alpha)) = \alpha + ((-\alpha) + \beta) = (\alpha + (-\alpha)) + \beta = 0 + \beta = \beta + 0 = \beta$ by definition.
    Therefore, $\alpha + (\beta - \alpha) = \beta$ holds.
    
  \item We have $\alpha0 + \alpha0 = \alpha(0 + 0) = \alpha0 = \alpha 0 + 0$ by definition,
    hence $\alpha0 = 0$ by Exercise 1(b).
    We also have $\alpha\cdot0 = 0\cdot\alpha$ by definition.
    Therefore, $\alpha\cdot0 = 0\cdot\alpha = 0$
    
  \item We have $\alpha + (-1)\alpha = 1\alpha + (-1)\alpha = (1 + (-1))\alpha = 0\alpha = 0$ by definition and Exercise 1(d).
      Since the additive inverse is unique,
      we obtain $(-1)\alpha = -\alpha$.
    
  \item We have $(-\alpha)(-\beta) = ((-1)\alpha)((-1)\beta) = (\alpha(-1))((-1)\beta) = \alpha((-1)((-1)\beta)) = \alpha((-1)(-1)\beta)$ by Exercise 1(e) and definition.
    We also have $(-1)(-1) = 0 + (-1)(-1) = (1 + (-1)) + (-1)(-1) = 1 + (-1) + (-1)(-1) = 1 + (-1)((-1) + 1) = 1 + (-1)(1 + (-1)) = 1 + (-1)0 = 1 + 0 = 1$ by definition.
    By it and definition, $\alpha((-1)(-1)\beta) = \alpha(1\beta) = \alpha(\beta1) = \alpha\beta$ holds.
    Therefore, $(-\alpha)(-\beta) = \alpha\beta$ holds.
    
  \item\label{it:ex1.1.g}
    If $\alpha\beta = 0$,
    suppose $\alpha \neq 0$ and $\beta \neq 0$ hold.
    By supposition and definition, we have $0 = \alpha^{-1}0 = \alpha^{-1}(\alpha\beta) = (\alpha^{-1}\alpha)\beta = (\alpha\alpha^{-1})\beta = 1\beta = \beta1 = \beta$,
    hence $\beta = 0$.
    However, this result contradicts supposition, ``$\alpha \neq 0$ and $\beta \neq 0$''.
    Therefore, if $\alpha\beta = 0$, then either $\alpha = 0$ or $\beta = 0$ (or both).
\end{enumerate}
\section{Mohehe (2022/09/19)}
\subsection{Exercise 1.2}
\begin{enumerate}[label = (\alph*)]
  \item The set of positive integers is not a field since there is no additive inverse for 1.
  \item The set of integers is not a field since there is no multiplicative inverse for 2.
  \item There exists a bijective map $\varphi$ from $\N$ (or $\Z$) to $\Q$ \cite{Q_countable}, where $\Q$ is a field \cite{Q_field}.
  	We can make $\N$ a field by re-defining (i) addition by $a \oplus b = \varphi^{-1} (\varphi(a) + \varphi(b))$ and (ii) multiplication by $a \otimes b = \varphi^{-1}(\varphi(a)\varphi(b))$ for each $a, b \in \N$.
    Note that the additive and multiplicative identities become $\varphi^{-1}(0)$ and $\varphi^{-1}(1)$, respectively.
    For each $\alpha\in\N$, the additive inverse becomes $\varphi^{-1}(-\varphi(\alpha))$, and the multiplicative inverse becomes $\varphi^{-1}(1/\varphi(\alpha))$ if $\alpha\ne\varphi^{-1}(0)$.
    
    Let $\alpha,\beta,\gamma,\alpha',\beta'\in\N$.
    Note that
    \begin{enumerate}[label = \arabic*)]
      \item $\alpha \oplus \beta = \varphi^{-1} (\varphi(\alpha) + \varphi(\beta)) = \varphi^{-1} (\varphi(\beta) + \varphi(\alpha)) = \beta \oplus \alpha$ holds.(addition is commutative)\\
      \item $\alpha \oplus (\beta \oplus \gamma) = \alpha \oplus (\varphi^{-1}(\varphi(\beta) + \varphi(\gamma))) = \varphi^{-1}(\varphi(\alpha) + \varphi(\varphi^{-1}(\varphi(\beta) + \varphi(\gamma))))  = \varphi^{-1}(\varphi(\alpha) + (\varphi(\beta) + \varphi(\gamma))) = \varphi^{-1}((\varphi(\alpha) + \varphi(\beta)) + \varphi(\gamma)) = \varphi^{-1}(\varphi(\varphi^{-1}(\varphi(\alpha) + \varphi(\beta))) + \varphi(\gamma)) = \varphi^{-1}(\varphi(\alpha) + \varphi(\beta)) \oplus \gamma = (\alpha \oplus \beta) \oplus \gamma$ holds.(addition is associative)
      \item $\alpha \oplus \varphi^{-1}(0) = \varphi^{-1}(\varphi(\alpha) + \varphi(\varphi^{-1}(0))) = \varphi^{-1}(\varphi(\alpha) + 0) = \varphi^{-1}(\varphi(\alpha)) = \alpha$ holds.(there exists additive identity, $\varphi^{-1}(0)$)
        If $\alpha'$ and $\beta'$ are additive identity, we have $\alpha' = \alpha' \oplus \beta' = \beta' \oplus \alpha' = \beta'$ by 1) and the definition of additive identity.(additive identity is unique)
      \item $-\varphi(\alpha)\in\Q$ holds by definition, so $\varphi^{-1}(-\varphi(\alpha)) \in \N$ holds. 
       Therefore, $\alpha \oplus \varphi^{-1}(-\varphi(\alpha)) = \varphi^{-1}(\varphi(\alpha) + \varphi(\varphi^{-1}(-\varphi(\alpha)))) = \varphi^{-1}(\varphi(\alpha) +  (-\varphi(\alpha))) = \varphi^{-1}(0)$ holds.(for each $\alpha$ ($\alpha\in\N$), there exists additive inverse)
        For each $\alpha$, if $\alpha'$ and $\beta'$ are additive inverse, we have $\alpha' = \alpha' \oplus \varphi^{-1}(0) = \alpha' \oplus (\alpha \oplus \beta') = (\alpha' \oplus \alpha) \oplus \beta' = (\alpha \oplus \alpha') \oplus \beta' = \varphi^{-1}(0) \oplus \beta' = \beta \oplus \varphi^{-1}(0) = \beta'$ by 1), 2), 3) and the definition of additive inverse.(additive inverse is unique)
      \item $\alpha \otimes \beta = \varphi^{-1}(\varphi(\alpha)\varphi(\beta)) = \varphi^{-1}(\varphi(\beta)\varphi(\alpha)) = \beta \otimes \alpha$ holds.(multiplication is commutative)
      \item $\alpha \otimes (\beta \otimes \gamma) = \alpha \otimes (\varphi^{-1}(\varphi(\beta)\varphi(\gamma))) = \varphi^{-1}(\varphi(\alpha)\varphi(\varphi^{-1}(\varphi(\beta)\varphi(\gamma)))) = \varphi^{-1}(\varphi(\alpha)\varphi(\beta)\varphi(\gamma)) = \varphi^{-1}(\varphi(\varphi^{-1}(\varphi(\alpha)\varphi(\beta)))\varphi(\gamma)) = \varphi^{-1}(\varphi(\alpha)\varphi(\beta)) \otimes \gamma = (\alpha \otimes \beta) \otimes \gamma$ holds.(multiplication is associative)
      \item $\alpha \otimes \varphi^{-1}(1) = \varphi^{-1}(\varphi(\alpha)\varphi(\varphi^{-1}(1))) = \varphi^{-1}(\varphi(\alpha)\cdot1) = \alpha$ holds.(there exists additive identity, $\varphi^{-1}(1)$)
        If $\alpha'$ and $\beta'$ are additive identity, we have $\alpha' = \alpha' \otimes \beta' = \beta' \otimes \alpha' = \beta'$ by 5) and definition of multiplicative identity.(multiplicative identity is unique)
      \item For each $\alpha$ $(\alpha \neq \varphi^{-1}(0))$, $(1/\varphi(\alpha))\in\Q$ holds by definition, so $\varphi^{-1}(1/\varphi(\alpha)) \in \N$ holds. Therefore, $\alpha \otimes \varphi^{-1}(1/\varphi(\alpha)) = \varphi^{-1}(\varphi(\alpha)\varphi(\varphi^{-1}(1/\varphi(\alpha)))) = \varphi^{-1}(\varphi(\alpha)(1/\varphi(\alpha))) = \varphi^{-1}(1)$ holds.(for each $\alpha$ ($\alpha \in \N$), there exists multiplicative inverse)
        For each $\alpha$ ($\alpha \neq \varphi^{-1}(0)$), if $\alpha'$ and $\beta'$ are multiplicative inverse, we have $\alpha' = \alpha' \otimes \varphi^{-1}(1) = \alpha' \otimes (\alpha \otimes \beta') = (\alpha' \otimes \alpha) \otimes \beta' = (\alpha \otimes \alpha') \otimes \beta' = \varphi^{-1}(1) \otimes \beta' = \beta' \otimes \varphi^{-1}(1) = \beta'$ by 5), 6), 7) and the definition of multiplicative inverse.(multiplicative inverse is unique) 
      \item $\alpha \otimes (\beta \oplus \gamma) = \alpha \otimes (\varphi^{-1}(\varphi(\beta) + \varphi(\gamma))) = \varphi^{-1}(\varphi(\alpha)\varphi(\varphi^{-1}(\varphi(\beta) + \varphi(\gamma)))) = \varphi^{-1}(\varphi(\alpha)(\varphi(\beta) + \varphi(\gamma))) = \varphi^{-1}(\varphi(\alpha)\varphi(\beta) + \varphi(\alpha)\varphi(\gamma)) = \varphi^{-1}(\varphi(\varphi^{-1}(\varphi(\alpha)\varphi(\beta))) + \varphi(\varphi^{-1}(\varphi(\alpha)\varphi(\gamma)))) = \varphi^{-1}(\varphi(\alpha \otimes \beta) + \varphi(\alpha \otimes \gamma)) = \alpha \otimes \beta \oplus \alpha \otimes \gamma$ holds.(distributive law stands)
    \end{enumerate}
\end{enumerate}

\section{Mohehe (2022/10/8)}
\subsection{Exercise 1.3}
\newcommand\rem{\mathbin\%} % 新しいコマンド \rem を定義.
\newcommand\Zm{\mathcal Z_m} % 新しいコマンド \Zm を定義.
For two integers $a$ and $b$,
we denote by $a \rem b$ the remainder after dividing $a$ by $b$,
and write $b \mid a$ if and only if $a \rem b = 0$.
For clarity,
we denote the ordinary sum and product of two integers $a$ and $b$
by $a +_\Z b$ and $a \cdot_\Z b$, respectively.
Note that $\alpha + \beta = (\alpha +_\Z \beta) \rem m$
and $\alpha\beta = (\alpha \cdot_\Z \beta) \rem m$
for $\alpha, \beta \in \Zm$.
\begin{enumerate}[label=(\alph*)]
  \item  Let $\alpha,\beta,\gamma \in \Zm$, $k \in \Z$
    \begin{enumerate}[label = \arabic*']
      \item\label{it:ex1.3.a.r} Proof : if $m$ is a prime, $\Zm$ is a field.\\
        Suppose $m$ is a prime,
        \begin{enumerate}[label=\arabic*)]
          \item\label{it:ex1.3.a.r.A1} $\alpha + \beta = (\alpha +_\Z \beta) \rem m = (\beta +_\Z \alpha) \rem m = \beta + \alpha$ (addition is commutative)
          \item  Since $\alpha +_\Z (\beta + \gamma) = \alpha +_\Z (\beta +_\Z \gamma) \rem m \equiv \alpha +_\Z (\beta +_\Z \gamma) = (\alpha +_\Z \beta) +_\Z \gamma \equiv (\alpha +_\Z \beta) \rem m +_\Z \gamma = (\alpha + \beta) +_\Z \gamma \pmod m$ holds, $\alpha + (\beta + \gamma) = (\alpha +_\Z (\beta + \gamma)) \rem m  = ((\alpha + \beta) +_\Z \gamma) \rem m = (\alpha + \beta) + \gamma $ holds.(addition is associative)
          \item  $\alpha + 0 = (\alpha +_\Z 0) \rem m = \alpha \rem m = \alpha$ (there exists additive identity)  By it and 1), if $\beta$ and $\gamma$ are additive identity, $\beta = \beta + \gamma = \gamma + \beta = \gamma$ (additive identity is unique)
          \item  If $\alpha + _\Z\beta = m$, $\alpha + \beta = (\alpha + _\Z\beta) \rem m = m \rem m = 0$ (there exists additive inverse)
          \item  $\alpha\beta = (\alpha \cdot_\Z \beta) \rem m = (\beta \cdot_\Z \alpha) \rem m = \beta\alpha$ (multiplication is commutative)
          \item  Since $\alpha \cdot_\Z (\beta\gamma) = \alpha \cdot_\Z ((\beta \cdot_\Z \gamma) \rem m) \equiv \alpha \cdot_\Z (\beta \cdot_\Z \gamma) = (\alpha \cdot_\Z \beta) \cdot_\Z \gamma \equiv ((\alpha \cdot_\Z \beta) \rem m) \cdot_\Z \gamma = (\alpha\beta) \cdot_\Z \gamma \pmod m$ holds, $\alpha(\beta\gamma) = (\alpha \cdot_\Z (\beta\gamma)) \rem m = ((\alpha\beta)\cdot_\Z \gamma) \rem m = (\alpha\beta)\gamma$ holds.(multiplication is associative)
          \item\label{it:ex1.3.a.r.B3} $\alpha1 = (\alpha _\Z1)\rem m = \alpha \rem m = \alpha$ (there exists multiplicative identity)
            By it and 5), if $\beta$ and $\gamma$ are multiplicative identity, $\beta = \beta\gamma = \gamma\beta = \gamma$ (multiplicative identity is unique)
          \item For all $\alpha (\alpha \neq 0)$, suppose there doesn't exist $\beta$ that makes $\alpha\beta = 1$.
            There exist $\beta, \gamma \in \Zm$ with $\beta \neq \gamma$ and $\alpha\beta = \alpha\gamma$, because $\beta$ is any one from 0 to $m-1$ and $\alpha\beta$ is any one from 0 to $m-1$ except 1.
            Therefore, $(\alpha \cdot_\Z \beta +_\Z (- \alpha \cdot_\Z \gamma) =)\; \alpha \cdot_\Z (\beta +_\Z (-\gamma)) = km$ holds.
            The right side has divisor m, but it contradicts that the left side doesn't have divisor of m except 1,  because $0<\alpha<(m-1)$ and ($(-m)<(\beta +_\Z (-\gamma))<0$ or $0<(\beta +_\Z (-\gamma))<m$) holds.
            Thus, there exists $\beta$ that makes $\alpha\beta = 1$.(there exists maltiplicative inverse)
            
            \color{red}
            A brief proof: Since each $\alpha\in\Zm\setminus\{0\}$ is coprime to $m$, there exist integers $x$ and $y$ such that $\alpha\cdot_\Z x +_\Z m\cdot_\Z y = 1$ by \cite{Bezout}.
            Putting $x' = x\rem m\in\Zm$, we obtain $\alpha x' = (\alpha \cdot_\Z x) \rem m = (\alpha\cdot_\Z x +_\Z m\cdot_\Z y) \rem m = 1 \rem m = 1$.
            Hence $x' = \alpha^{-1}$.
            \color{black}
          \item\label{it:ex1.3.a.r.C} $\alpha(\beta + \gamma) = (\alpha \cdot_\Z (\beta + \gamma))\rem m = (\alpha \cdot_\Z ((\beta +_\Z \gamma)\rem m))\rem m \equiv (\alpha \cdot_\Z (\beta +_\Z \gamma))\rem m = (\alpha \cdot_\Z \beta + \alpha \cdot_\Z \gamma)\rem m \equiv ((\alpha \cdot_\Z \beta)\rem m +_\Z (\alpha \cdot_\Z \gamma)\rem m)\rem m \equiv (\alpha \cdot_\Z \beta)\rem m + (\alpha \cdot_\Z \gamma)\rem m \equiv \alpha\beta + \alpha\gamma$ holds.(distributive law stands)
        \end{enumerate}
        In conclusion, if $m$ is a prime, $\Zm$ is a field.
      \item\label{it:ex1.3.a.l} Proof : If $\Zm$ is a field, $m$ is a prime.\\
        By contraposition, it is equivalent to prove ``If $m$ is not a prime, $\Zm$ is not a field.''
        We can show \ref{it:ex1.3.a.r.A1} to \ref{it:ex1.3.a.r.B3} and \ref{it:ex1.3.a.r.C} in the same way as \ref{it:ex1.3.a.r}.
        For each $m$, suppose there exist $\alpha$ and $\beta$ that make $\alpha\beta = 1$.
        $m$ is not a prime, so let $p$ be one of prime factors of $m$ and then we have $m = p \cdot_\Z p'$.($p' \in \Z$ and $1 < p' < m$)\\
        If $\alpha = p$, by $\alpha\beta = 1$ and $m = p \cdot_\Z p'$, we have $\alpha \cdot_\Z \beta = k \cdot_\Z m + 1  (k \in \Z) \Leftrightarrow p \cdot_\Z \beta = k \cdot_\Z p \cdot_\Z p' +_\Z 1 \Leftrightarrow (\beta +_\Z (-k \cdot_\Z p')) \cdot_\Z p = 1$.
        The right side is 1 but the left one is not 1 because of $1 < p$ and $\beta +_\Z (-k \cdot_\Z p') \in \Z$.
        Therefore It is contradicted.
        For each $m$, there doesn't exist $\alpha$ and $\beta$ that make $\alpha\beta = 1$.
        In conclusion, ``If $m$ is not a prime, $\Zm$ is not a field.'' and ``If $\Zm$ is a field, $m$ is a prime.''
    \end{enumerate}
    Because of \ref{it:ex1.3.a.r} and \ref{it:ex1.3.a.l}, $\Zm$ is a field if and only if $m$ is a prime.

  \item $4$
  \item $5$
\end{enumerate}

% \rem と \Zm はこの問題でしか使わないので,以下で消す
\let\rem\undefined
\let\Zm\undefined

\subsection{Exercise 1.4}
\newcommand\F{\mathfrak F}
Define $\alpha_n = \overbrace{1 + \cdots + 1}^\text{$n$ terms}$ for $n \in \{1, 2, \ldots\}$.
Then, \begin{align*}
  \alpha_m\alpha_n
  &= \alpha_m(\overbrace{1 + \dotsb + 1}^\text{$n$ terms}) \\
  &= \alpha_m((\overbrace{1+ \dotsb + 1}^\text{$(n-1)$ terms}) + 1)\\
  &= \alpha_m(\overbrace{1 + \dotsb + 1}^\text{$(n-1)$ terms}) + \alpha_m \cdot 1 && \text{(by distributive law)}\\
  &= \alpha_m(\overbrace{1 + \dotsb + 1}^\text{$(n-1)$ terms}) + \alpha_m && \text{(by definition of multiplicative identity)}\\
  &= \dotsb \\
  &= \alpha_m(1 + 1) + \overbrace{\alpha_m + \dotsb + \alpha_m}^\text{$(n-2)$ terms}\\
  &= \alpha_m \cdot 1 + \alpha_m \cdot 1 + \overbrace{\alpha_m + \dotsb + \alpha_m}^\text{$(n-2)$ terms} && \text{(by distributive law)} \\
  &= \overbrace{\alpha_m + \dotsb + \alpha_m}^\text{$n$ terms} && \text{(by definition of multiplicative identity)} \\
  &= \overbrace{1 + \dotsb + 1}^\text{$m$ terms} + (\overbrace{1 + \dotsb + 1}^\text{$m$ terms}) + \overbrace{\alpha_m + \dotsb + \alpha_m}^\text{$(n-2)$ terms} \\
  &= \overbrace{1 + \dotsb + 1}^\text{$m$ terms} + ((\overbrace{1 + \dotsb + 1}^\text{$(m-1)$ terms}) + 1) + \overbrace{\alpha_m + \dotsb + \alpha_m}^\text{$(n-2)$ terms} \\
  &= \overbrace{1 + \dotsb + 1}^\text{$m$ terms} + (1 + (\overbrace{1 + \dotsb + 1}^\text{$(m-1)$ terms})) + \overbrace{\alpha_m + \dotsb + \alpha_m}^\text{$(n-2)$ terms} && \text{(by commutative property)}\\
  &= \overbrace{1 + \dotsb + 1}^\text{$(m+1)$ terms} + (\overbrace{1 + \dotsb + 1}^\text{$(m-1)$ terms}) + \overbrace{\alpha_m + \dotsb + \alpha_m}^\text{$(n-2)$ terms} && \text{(by associative property)}  \\
  &= \dotsb \\
  &= \overbrace{1 + \dotsb + 1}^\text{$(2m-2)$ terms} + (1 + 1) + \overbrace{\alpha_m + \dotsb + \alpha_m}^\text{$(n-2)$ terms} \\
  &= \overbrace{1 + \dotsb + 1}^\text{$2m$ terms} + \overbrace{\alpha_m + \dotsb + \alpha_m}^\text{$(n-2)$ terms} && \text{(by associative property)} \\
  &= \dotsb \\
  &= \overbrace{1 + \dotsb + 1}^\text{$m(n-1)$ terms} + \alpha_m \\
  &= \dotsb \\
  &= \overbrace{1 + \dotsb + 1}^\text{$(mn-2)$ terms} + (1 + 1) \\ 
  &= \overbrace{1 + \dotsb + 1}^\text{$mn$ terms} && \text{(by associative property)}\\
  &= \alpha_{mn}
\end{align*} for all $m, n$.\\\\
Assume there exists an $n$ with $\alpha_n = 0$ but $\alpha_k \ne 0$ for any $k < n$.
It suffices to prove that $n$ is a prime.\\
% ここに証明\UTF{2461}%
Suppose n is not any prime. Let p be one of prime factors of $n$ and then we have $n = p p' (p' \in\N$ and $p' > 1)$.
By $\alpha_m \alpha_n = \alpha_{mn}$ for all $m$ and $n$, $\alpha_n = \alpha_p \alpha_{p'}$ holds.
We have either $\alpha_p = 0$ or $\alpha_{p'} = 0$ (or both) because of $\alpha_n = 0$ and \nameref{sec:ex1.1} \ref{it:ex1.1.g}.
However, it is contradictory to $\alpha_p = 0$ and $\alpha_{p'} = 0$. Therefore, $n$ is a prime.

\subsection{Exercise 1.5} 
\begin{enumerate}[label = (\alph*)]
\item
For the followings, it is used that $\Q$ and $\R$ are fields. Note that $\sqrt2 \in \R$ and $\sqrt2 \not\in \Q$\\
Let $\alpha_1, \alpha_2, \alpha_3 \in \Q(\sqrt2) \subset \R$.\\
For all $\alpha_1, \alpha_2 \in \Q(\sqrt2)$, $\alpha_1 + \alpha_2 \in \Q(\sqrt2)$ and $\alpha_1\alpha_2 \in \Q(\sqrt2)$ by the followings.\\
There exist $a,b,c,d \in \Q$, $\alpha_1 = a + b\sqrt2$ and $\alpha_2 = c + d\sqrt2$ hold.\\
We have $\alpha_1 + \alpha_2 = (a + b\sqrt2) + (c + d\sqrt2) = (a + c) + (b + d)\sqrt2$ and $(a + c),(b + d) \in \Q$, so $\alpha_1 + \alpha_2 \in \Q(\sqrt2)$.\\
In addition, we have $\alpha_1\alpha_2 = (a + b\sqrt2)(c + d\sqrt2) = (ac + 2bd) + (ad + bc)\sqrt2$ and $(ac + 2bd),(ad +bc) \in \Q$, so $\alpha_1\alpha_2 \in \Q(\sqrt2)$
\begin{enumerate}[label=\arabic*)]
\item $\alpha_1 + \alpha_2 = \alpha_2 + \alpha_1$ (addition is commutative)
\item $\alpha_1 + (\alpha_2 + \alpha_3) = (\alpha_1 + \alpha_2) + \alpha_3$ (addition is associative)
\item We have $0 = 0 + 0\sqrt2 \in \Q(\sqrt2)$ and $\alpha_1 + 0 = \alpha_1$\\($\Q(\sqrt2)$ has additive identity)
\item For all $\alpha_1 \in \Q(\sqrt2)$, put $\alpha_1 = a + b\sqrt2$ with $a,b \in \Q$.
There exists $\alpha'_1 \in \Q(\sqrt2)$ with $\alpha'_1 = (-a) + (-b)\sqrt2$. 
We have $\alpha_1 + \alpha'_1 = a + b\sqrt2 + (-a) + (- b)\sqrt2 = a - a + b\sqrt2 - b\sqrt2 = 0$. 
Therefore, to every $\alpha_1 \in \Q(\sqrt2)$, there corresponds $\alpha'_1 \in \Q(\sqrt2)$ with $\alpha_1 + (-\alpha_1) = 0$
\item $\alpha_1\alpha_2 = \alpha_2\alpha_1$ (multiplication is commutative)
\item $\alpha_1(\alpha_2\alpha_3) = (\alpha_1\alpha_2)\alpha_3$ (multiplication is associative)
\item We have $1 = 1 + 0 \sqrt2 \in \Q(\sqrt2)$ and $\alpha_1 \cdot 1 = \alpha_1$\\
($\Q(\sqrt2)$ has multiplicative identity)
\item For all $\alpha_1 \in \Q(\sqrt2)$ with $\alpha_1 \ne 0$, put $\alpha_1 = a + b\sqrt2$ with $a,b \in \Q$. In this case, $a \ne 0$ or $b \ne 0$ holds by the followings.\\
``If $\alpha_1 = 0$, we have $\alpha_1 = a + b\sqrt2 = 0 \Leftrightarrow a = -b\sqrt2$. 
Therefore, $a = b = 0$ by $a,b \in \Q$.''\\
Let $\alpha''_1 = \dfrac{a}{a^2 - 2b^2} + \left(-\dfrac{b}{a^2 - 2b^2}\right)\sqrt2 \in \Q(\sqrt2)$. Note that we have $a^2 -2b^2 = (a + b\sqrt2)(a - b\sqrt2)$ and $a,b \in \Q$ with ($a \ne 0$ or $b \ne 0$), so we have $a + b\sqrt2 \ne 0$ and $a - b\sqrt2 \ne 0$, and then $a^2 -2b^2 \in \Q$ with $a^2 -2b^2 \ne 0$.
We have $\alpha_1\alpha''_1 = (a + b\sqrt2)\left(\dfrac{a}{a^2 - 2b^2} + \left(-\dfrac{b}{a^2 - 2b^2}\right)\sqrt2\right) = \dfrac{a^2 - ab\sqrt2 + ab\sqrt2 - 2b^2 }{a^2 - 2b^2} = 1$. 
Therefore, to every $\alpha_1 \in \Q(\sqrt2)$ with $\alpha_1 \ne 0$, there exists $\alpha''_1 \in \Q(\sqrt2)$ with $\alpha_1\alpha''_1 = 1$
\item $\alpha_1(\alpha_2 + \alpha_3) = \alpha_1\alpha_2 + \alpha_1\alpha_3$ (distributive law stands)
\end{enumerate}
from 1) to )9, $\Q(\sqrt2)$ is a field.
\item
Let $\Z(\sqrt2)$ be the set of all numbers of the form $\alpha + \beta\sqrt2$, where $\alpha$ and $\beta$ are integers. If $\Z(\sqrt2)$ is a field, $2 = 2 + 0\sqrt2 (\in \Z(\sqrt2))$ has multiplicative inverse. There exists $\exists\beta_1 = \{\alpha + \beta\sqrt2 |\beta_1 \in \Z(\sqrt2)\}$ with $2\beta_1 = 1 \iff \beta_1 = \frac{1}{2}$. However, $\frac{1}{2} \not\in \Z(\sqrt2)$ holds, so $\Z(\sqrt2)$ is not a field.\\\\
Another way : Let $\Z(\sqrt2)$ be the set of all numbers of the form $\alpha + \beta\sqrt2$, where $\alpha$ and $\beta$ are integers. 
$\Z(\sqrt2)$ is not a field since there is no multiplicative inverse for $2 + \sqrt2 \in \Z(\sqrt2)$ by the followings.\\
Suppose there exists multiplicative inverse for $2 + \sqrt2$.
There exists $\exists\beta_1 \in \Z(\sqrt2)$ with $\beta_1 = \alpha + \beta\sqrt2$ and $(2 + \sqrt2)\beta_1 = 1$ by supposition.
Therefore, we have $(2 + \sqrt2)\beta_1 = (2 + \sqrt2)(\alpha + \beta\sqrt2) = 2(\alpha + \beta) + (\alpha + 2\beta)\sqrt2 = 1 \iff 2(\alpha + \beta) - 1 = -(\alpha + 2\beta)\sqrt2 \implies 2(2(\alpha + \beta)^2 - 2(\alpha + \beta)) - 1 = 2(\alpha + 2\beta)^2$.
It is contradicted because the left is odd number and the right is even number.
Therefore, $(2 + \sqrt2)\beta_1 = 1$ is contradicted and then there is no multiplicative inverse for $2 + \sqrt2$.
\end{enumerate}

\subsection{Exercise 1.6}
\begin{enumerate}[label = (\alph*)]
\item
Let $P$ be such set of all polynomials with integer coefficients, %Pは関数
$\id \in P$ be $\id(x) = x$ ($x \in \R$), and 
$I \in P$ be $I(x) = 1$ ($x \in \R$).

Suppose there exists $q \in P$ with ${\id} \cdot q = I$.
Then, we have $\id(0) \cdot q(0) = 0$, and it is contradicted to supposition.
Therefore, there does not exist  $q \in P$ with ${\id} \cdot q = I$. In other words, $\id$ does not have the multiplicative inverse.
In conclusion, the set of all polynomials with integer coefficients does not form a field.   
\item
the set of all polynomials with real number coefficients does not form a field for the same reason.   
\end{enumerate}

\subsection{Exercise 1.7}
\begin{enumerate}[label = (\alph*)]
\item
Suppose $\F$ is a field.
Let $(\alpha,\beta) \in \F$ with $\alpha,\beta \in \R$. 
Then, additive identity would be $(0,0)$, because for all $\alpha,\beta$, we have $\alpha + 0 = \alpha, \beta + 0 = \beta$.
In additon, multiplicative identity would be $(1,1)$, because for all $\alpha,\beta$, we have $\alpha1 = \alpha, \beta1 = \beta$.

Here, think about $(0,1) (\ne (0,0))$.
For all $(\alpha, \beta)$, we have $(0,1)(\alpha, \beta) = (0, \beta) (\ne (1,1))$. Therefore, $(0,1)$ does not have multiplicative inverse.
In conclusion, $\F$ is not a field. 

\item
Let $(\alpha_1,\beta_1),(\alpha_2,\beta_2),(\alpha_3,\beta_3) \in \F$ with $\alpha_1,\alpha_2,\alpha_3,\beta_1,\beta_2,\beta_3 \in \R$.
\begin{enumerate}[label=(\Alph*)]
\item
\begin{enumerate}[label=(\arabic*)]
\item $(\alpha_1,\beta_1) + (\alpha_2,\beta_2) = (\alpha_1 + \alpha_2, \beta_1 + \beta_2) = (\alpha_2, \beta_2) + (\alpha_1, \beta_1)$.(addition is commutative)
\item $(\alpha_1, \alpha_1) + ((\alpha_2,\beta_2) + (\alpha_3, \beta_3)) = (\alpha_1 + \alpha_2 + \alpha_3, \beta_1 + \beta_2 + \beta_3) = ((\alpha_1,\beta_1) + (\alpha_2, \beta_2)) + (\alpha_3, \beta_3)$.(addition is associative)
\item There exists $(0,0) \in \F$ such that $(\alpha_1, \beta_1) + (0,0) = (\alpha_1, \beta_1)$ for every $(\alpha_1,\beta_1)$. ($\F$ has additive identity)
\item For every $(\alpha_1,\beta_1)$, there exists $(-\alpha_1, -\beta_1) \in \F$ such that $(\alpha_1, \beta_1) + (-\alpha_1, -\beta_1) = (0,0)$.
\end{enumerate}
\item
\begin{enumerate}[label=(\arabic*)]
\item $(\alpha_1,\beta_1)(\alpha_2,\beta_2) = (\alpha_1\alpha_2 - \beta_1\beta_2,\alpha_1\beta_2 + \alpha_2\beta_1) = (\alpha_2,\beta_2)(\alpha_1,\beta_1)$. (multiplication is commutative)
\item $((\alpha_1,\beta_1)(\alpha_2,\beta_2))(\alpha_3,\beta_3) = (\alpha_1\alpha_2\alpha_3-\beta_1\beta_2\alpha_3-\alpha_1\beta_2\beta_3-\beta_1\alpha_2\beta_3,\alpha_1\beta_2\alpha_3+\beta_1\alpha_2\alpha_3+\alpha_1\alpha_2\beta_3-\beta_1\beta_2\beta_3) = (\alpha_1,\beta_1)((\alpha_2,\beta_2)(\alpha_3,\beta_3))$.(multiplication is associative)
\item There exists $(1,0) \in \F$ such that $(\alpha_1,\beta_1)(1,0) = (\alpha_1,\beta_1)$ for every $(\alpha_1,\beta_1)$. ($\F$ has multiplicative identity)
\item For every $(\alpha_1,\beta_1) (\ne (0,0))$, there exists $\left(\frac{\alpha_1}{\alpha_1^2+\beta_1^2},-\frac{\beta_1}{\alpha_1^2+\beta_1^2}\right) \in \F$ such that $(\alpha_1,\beta_1)\left(\frac{\alpha_1}{\alpha_1^2+\beta_1^2},-\frac{\beta_1}{\alpha_1^2+\beta_1^2}\right) = (1,0)$.
\end{enumerate}
\item $(\alpha_1,\beta_1)((\alpha_2,\beta_2) + (\alpha_3,\beta_3)) = (\alpha_1,\beta_1)(\alpha_2+\alpha_3,\beta_2+\beta_3) = (\alpha_1\alpha_2 + \alpha_1\alpha_3-\beta_1\beta_2-\beta_1\beta_3,\alpha_1\beta_2+\alpha_1\beta_3+\beta_1\alpha_2+\beta_1\alpha_3) = (\alpha_1\alpha_2 - \beta_1\beta_2,\alpha_1\beta_2 + \beta_1\alpha_2) + (\alpha_1\alpha_3 - \beta_1\beta_3,\alpha_1\beta_3 + \alpha_3\beta_1) = (\alpha_1,\beta_1)(\alpha_2,\beta_2) + (\alpha_1,\beta_1)(\alpha_3,\beta_3)$.(distributive law stands)
\end{enumerate}
\item
Let $\F'$ be the set of all pairs of $(\alpha,\beta)$ of complex numbers.
\begin{itemize}
\item[(a)]
Suppose $\F'$ is a field.
Let $(\alpha,\beta) \in \F'$ with $\alpha,\beta \in \C$. 
Then, additive identity would be $(0,0)$, because for all $\alpha,\beta$, we have $\alpha + 0 = \alpha, \beta + 0 = \beta$.
In additon, multiplicative identity would be $(1,1)$, because for all $\alpha,\beta$, we have $\alpha1 = \alpha, \beta1 = \beta$.

Here, think about $(0,1) (\ne (0,0))$.
For all $(\alpha, \beta)$, we have $(0,1)(\alpha, \beta) = (0, \beta) (\ne (1,1))$. Therefore, $(0,1)$ does not have multiplicative inverse.
In conclusion, $\F'$ is not a field. 
\item[(b)]
Suppose $\F'$ is a field.
Let $(\alpha,\beta) \in \F'$ with $\alpha, \beta \in \C$. 
Then, additive identity would be $(0,0)$, because for all $\alpha,\beta$, we have $\alpha + 0 = \alpha, \beta + 0 = \beta$.
In additon, multiplicative identity would be $(1,0)$, because for all $\alpha,\beta$, we have $\alpha1 - \beta0 = \alpha, \alpha0 + \beta1 = \beta$.

Here, think about $(i,1) (\ne (0,0))$.
There exists $(\alpha, \beta)$ such that $(i,1)(\alpha, \beta) = (\alpha i -\beta, \beta i + \alpha) = (1,0)$. 
Then, we have $\alpha i -\beta = 1$,and $\beta i + \alpha = 0 \iff \alpha i -\beta = 0$. It is contradicted.
Therefore, $(i,1)$ does not have multiplicative inverse.
In conclusion, $\F'$ is not a field. 
\end{itemize}
\end{enumerate}

\section{Johno (2022/11/27)}
\subsection{Exercise 4.1}
\begin{enumerate}[label = (\alph*)]
\item We have $0+x = x+0 = x$ by definition.
\item It follows from definition that $0+0=0$, hence $0 = -0$ by the uniqueness of additive inverse.
\item We can prove this as in Exercise 1.1 (d).
\item The same as above.
\item Let $\alpha x = 0$ hold.
If $\alpha \ne 0$, then $x = 1x = (\frac{1}{\alpha}\alpha)x = \frac{1}{\alpha} (\alpha x) = \frac{1}{\alpha}0 = 0$ by definition and Exercise 2.1 (c).
\item By definition and Exercise 2.1 (d), we have $1x + (-1)x = (1-1)x  = 0x =0$.
Hence $-x = -(1x) = (-1)x$.
\item It follows from definition that $y + (x-y) = y + (-y + x) = (y - y) + x = 0 + x =x + 0 = x$.
\end{enumerate}

\section{Mohehe (2022/11/31)}
\newcommand \V{\mathcal V}
\newcommand \PP{\mathcal P}
\subsection{Exercise 4.2}
We have $Z_p^n = \{(x_1,x_2, \dots , x_n):x_1,x_2,\dots,x_n \in Z_p\}$.
Moreover, $Z_p$ has $p$ members.
Therefore, the number of  the vectors in this vector space is $p^n$.

\subsection{Exercise 4.3}
Suppose $\V$ is a vector space.
If $x = (0,1) \in \V$, we have $1(0,1) = (0,1)$ by definition of vector spaces, but we also have$1(0,1) = (0,0)$ by definition of $\V$. It is contradicted, so $\V$ is not a vector space.

\subsection{Exercise 4.4}
\begin{enumerate}[label=(\alph*)]
\item
For $(1,0,0) \in\V$ and $i \in\C$, we have $i(1,0,0) = (i,0,0) \not\in\V$, because $i \not\in\R$. Therefore $\V$ is not a vector space.
\item
Let $(0,a_2,a_3), (0,b_2,b_3), (0,c_2,c_3) \in\V$ and $\alpha, \beta \in\C$.

We have $(0,a_2,a_3) + (0,b_2,b_3) = (0,a_2+b_2,a_3+b_3) \in\V$ and $\alpha(0,a_2,a_3) = (0,\alpha a_2,\alpha a_3) \in\V$. Therefore $\V$ is closed under addition and scalar multiplication.
\begin{enumerate}[label = (\Alph*)]
\item
\begin{enumerate}[label = (\arabic*)]
\item
$(0,a_2,a_3) + (0,b_2,b_3) = (0,a_2+b_2,a_3+b_3) = (0,b_2,b_3) + (0,a_2,a_3)$. (addition is commutative)
\item
$(0,a_2,a_3)+((0,b_2,b_3)+(0,c_2,c_3))=(0,a_2,a_3)+(0,b_2+c_2,b_3+c_3)=(0,a_2+b_2+c_2,a_3+b_3+c_3) = (0,a_2,a_3)+(0,b_2,b_3)+(0,c_2,c_3)$. (addition is associative)
\item
There exists $(0,0,0)\in\V$ such that $(0,a_2,a_3) + (0,0,0) = (0,a_2,a_3)$ for every $(0,a_2,a_3)$. ($\V$ has additive identity)
\item
For every $(0,a_2,a_3)$, there exists $(0,-a_2,-a_3)$ such that $(0,a_2,a_3) + (0,-a_2,-a_3) = (0,0,0)$.
\end{enumerate}
\item
\begin{enumerate}[label = (\arabic*)]
\item
$\alpha(\beta(0,a_2,a_3)) = (0,\alpha\beta a_2,\alpha\beta a_3) = (\alpha\beta)(0,a_2,a_3)$. (multiplication by scalars is associative)
\item
We have $1(0,a_2,a_3) = (0,a_2,a_3)$ for $1 \in\C$ and for every $(0,a_2,a_3)$. 
\end{enumerate}
\item
\begin{enumerate}[label = (\arabic*)]
\item
$\alpha((0,a_2,a_3) + (0,b_2,b_3)) = \alpha(0,a_2+b_2,a_3+b_3) = (0,\alpha a_2+\alpha b_2,\alpha a_3+\alpha b_3) = (0,\alpha a_2,\alpha a_3) + (0,\alpha b_2,\alpha b_3) = \alpha(0,a_2,a_3)+\alpha(0,b_2,b_3)$. 
\item
$(\alpha + \beta)(0,a_2,a_3) = (0,(\alpha+\beta)a_2,(\alpha+\beta)a_3) = (0,\alpha a_2+\beta a_2,\alpha a_3+\beta a_3) = (0,\alpha a_2,\alpha a_3) + (0,\beta a_2,\beta a_3) = \alpha(0,a_2,a_3) + \beta(0,a_2,a_3)$.
\end{enumerate}
\end{enumerate}
In conclusion, $\V$ is a vector space.
\item
For $(0,1,1),(1,0,1) \in \V$, we have $(0,1,1) + (1,0,1) = (1,1,2) \not\in\V$, because $1 \ne 0$. Therefore $\V$ is not a vector space.

\item

Let $(a_1,a_2,a_3), (b_1,b_2,b_3), (c_1,c_2,c_3) \in\V$ with $a_1+a_2=b_1+b_2=c_1+c_2=0$ and $\alpha, \beta \in\C$.

We have $(a_1,a_2,a_3) + (b_1,b_2,b_3) = (a_1+b_1,a_2+b_2,a_3+b_3) \in\V$ because $a_1+b_1+a_2+b_2=0$. Moreover, we have $\alpha(a_1,a_2,a_3) = (\alpha a_1,\alpha a_2,\alpha a_3) \in\V$ because $\alpha a_1+\alpha a_2=\alpha(a_1+a_2)=0$. Therefore $\V$ is closed under addition and scalar multiplication.
\begin{enumerate}[label = (\Alph*)]
\item
\begin{enumerate}[label = (\arabic*)]
\item
$(a_1,a_2,a_3) + (b_1,b_2,b_3) = (a_1+b_1,a_2+b_2,a_3+b_3) = (b_1,b_2,b_3) + (a_1,a_2,a_3)$. (addition is commutative)
\item
$(a_1,a_2,a_3)+((b_1,b_2,b_3)+(c_1,c_2,c_3))=(a_1,a_2,a_3)+(b_1+b_1,b_2+c_2,b_3+c_3)=(a_1+b_1+c_1,a_2+b_2+c_2,a_3+b_3+c_3) = (a_1,a_2,a_3)+(b_1,b_2,b_3)+(c_1,c_2,c_3)$. (addition is associative)
\item
There exists $(0,0,0)\in\V$ such that $(a_1,a_2,a_3) + (0,0,0) = (a_1,a_2,a_3)$ for every $(a_1,a_2,a_3)$. ($\V$ has additive identity)
\item
For every $(a_1,a_2,a_3)$, there exists $(-a_1,-a_2,-a_3)$ such that $(a_1,a_2,a_3) + (-a_1,-a_2,-a_3) = (0,0,0)$.
\end{enumerate}
\item
\begin{enumerate}[label = (\arabic*)]
\item
$\alpha(\beta(a_1,a_2,a_3)) = (\alpha\beta a_1,\alpha\beta a_2,\alpha\beta a_3) = (\alpha\beta)(a_1,a_2,a_3)$. (multiplication by scalars is associative)
\item
We have $1(a_1,a_2,a_3) = (a_1,a_2,a_3)$ for $1 \in\C$ and for every $(a_1,a_2,a_3)$. 
\end{enumerate}
\item
\begin{enumerate}[label = (\arabic*)]
\item
$\alpha((a_1,a_2,a_3) + (b_1,b_2,b_3)) = \alpha(a_1+b_1,a_2+b_2,a_3+b_3) = (\alpha a_1+\alpha b_1,\alpha a_2+\alpha b_2,\alpha a_3+\alpha b_3) = (\alpha a_1,\alpha a_2,\alpha a_3) + (\alpha b_1,\alpha b_2,\alpha b_3) = \alpha(a_1,a_2,a_3)+\alpha(b_1,b_2,b_3)$. 
\item
$(\alpha + \beta)(a_1,a_2,a_3) = ((\alpha+\beta)a_1,(\alpha+\beta)a_2,(\alpha+\beta)a_3) = (\alpha a_1+ \beta a_1,\alpha a_2+\beta a_2,\alpha a_3+\beta a_3) = (\alpha a_1,\alpha a_2,\alpha a_3) + (\beta a_1,\beta a_2,\beta a_3) = \alpha(a_1,a_2,a_3) + \beta(a_1,a_2,a_3)$.
\end{enumerate}
\end{enumerate}

\item
For $(1,0,0),(0,1,0) \in \V$, we have $(1,0,0) + (0,1,0) = (1,1,0) \not\in\V$, because $1 + 1 \ne 0$. Therefore $\V$ is not a vector space.
\end{enumerate}

\subsection{Exercise 4.5}
\begin{enumerate}[label = (\alph*)]
\item
For $t^3+t^2,-t^3 \in\V$, we have $t^3+t^2+(-t^3) = t^2 \not\in\V$, because $t^2$ doesn't have degree $3$. Therefore $\V$ is not a vector space.
\item
Let $x,y \in\V$, $\alpha \in \C$.

%(x+y)(0) = x(0)+y(0) はOK?
For $\V$ multiplication by scalars is distributive with respect to vector addition, because $\V\subset\PP$.
By it and $2x(0) = x(1)$, $2(x+y)(0) = 2(x(0)+y(0)) = 2x(0) + 2y(0) = x(1) + y(1) = (x+y)(1)$.
For $\V$ multiplication by scalars is associative, because $\V\subset\PP$.
By it and $2x(0) = x(1)$, $2(\alpha x)(0) = 2\alpha x(0) = \alpha 2x(0) = \alpha (2x(0)) = \alpha x(1)$
Therefore, $\V$ is closed.
For $\V$ addition is associative and commutative, multiplication by scalar is associative, $1x = x$ ($1 \in\C$) for every vector $x$, multiplication by scalars is distributive with respect to vector addition, and multiplication by vectors is distributive with respect to scalar addition, because $\V\subset\PP$.
In addition, there exists $0 \in\V$ such that $x + 0 = x$, and to every vector $x \in\V$ there corresponds a vector $-x \in\V$ such that $x + (-x) = 0$.
In conclution, $\V$ is vector space.

\item
Let $x \in\V$ with $x \ne 0$.
For $-1 \in\C$, $-1x = -x (\ne 0)$ by Exercise 2.1(f). However $-x \not\in\V$ because $-x < 0$.
Therefore $\V$ is not a vector space.
\item
Let $x,y \in\V$, $\alpha \in\C$.

We have $(x+y)(t) = x(t) + y(t) = x(1-t) + y(1-t) = (x+y)(1-t)$ and $(\alpha x)(t) = \alpha x(t) = \alpha x(1-t) = (\alpha t)(1-t)$ by $x(t) = x(1-t)$.
Therefore $\V$ is closed.
For $\V$, there exists $0 \in\V$ such that $x + 0 = x$, and to every vector $x \in\V$ there corresponds a vector $-x \in\V$ such that $x + (-x) = 0$.
Thus, similar to (d), $\V$ is vector space.
\end{enumerate}
\let\V\undefined
\let\PP\undefined

\section{Johno (2022/12/26)}
\subsection{Section 6}
\begin{remark}
If $\{x_i\}$ is linearly independent, then a necessary and sufficient condition that $x$ be a linear combination of $\{x_i\}$ is that the enlarged set, obtained by adjoining $x$ to $\{x_i\}$, be linearly dependent.
\end{remark}
\begin{proof}
%Since the statement obviously holds if $x=0$, we only consider the case $x\ne 0$.
Suppose $x = \sum_i\alpha_i x_i$.
Then $\sum_i(-\alpha_i) x_i + x = 0$, hence $\{x_i\}\cup \{x\}$ is dependent.
Next, suppose $\{x_i\}\cup \{x\}$ is dependent.
Then there exist a set $\{\alpha_i\}\cup\{\beta\}$ of scalars (not all zero) such that $\sum_i\alpha_i x_i + \beta x = 0$.
By the independence of $\{x_i\}$, $\beta = 0$ implies that $\alpha_i = 0$ for all $i$, a contradiction.
Hence $\beta \ne 0$ and we obtain $x = \sum_i (-\alpha_i/\beta)x_i$.
\end{proof}

\subsection{Exercise 7.1}
\begin{enumerate}[label = (\alph*)]
\item 
Since $x+y+z-u=0$, the four vectors are dependent. 
It is obvious that $\{x, y, z\}$ is independent.
Now let us consider the set $\{x,y,u\}$.
If $\alpha x + \beta y + \gamma u = 0$, then the three equations: $\alpha + \gamma = 0$, $\beta + \gamma = 0$, $\gamma = 0$ hold, so that $\alpha=\beta=\gamma=0$.
The independence of $\{x,z,u\}$ and $\{y,z,u\}$ can be similarly proved.
\item
We can prove this in a similar way as in Exercise 7.1(a).
\end{enumerate}

\section{Mohehe (2022/1/02)}
\subsection{Exercise 7.2}
Suppose when $\xi$ be rational, vectors $1$ and $\xi\in\R$ be linearly independent.
Let $\alpha\ne0\in\R$, then $\alpha\xi \in\R$.
$-\alpha\xi\cdot1+\alpha\cdot\xi=0$ so vectors $1$ and $\xi$ be linearly dependent. It is contradicted to supposition, so $\xi$ needs to be irrational.
\subsection{Exercise 7.3}
Let $\alpha,\beta, \text{ and }\gamma$ are scalars.
$\alpha(x+y)+\beta(y+z)+\gamma(z+x)=0 \iff (\gamma+\alpha)x+(\alpha+\beta)y+(\beta+\gamma)z=0 \iff \gamma+\alpha=\alpha+\beta=\beta+\gamma=0$ because $x,y,\text{ and }z$ are linearly independent. In addition, $\gamma+\alpha=\alpha+\beta=\beta+\gamma=0 \iff \alpha=\beta=\gamma=0$ so $x+y,y+x,\text{ and }z+x$ are linearly independent.

\begin{thebibliography}{9}
\bibitem{Q_countable} \url{https://proofwiki.org/wiki/Rational_Numbers_are_Countably_Infinite}
\bibitem{Q_field} \url{https://proofwiki.org/wiki/Rational_Numbers_form_Field}
\bibitem{Bezout} \url{https://proofwiki.org/wiki/Bezout\%27s_Identity}
\end{thebibliography}
\end{document}
